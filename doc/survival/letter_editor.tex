\documentclass{letter}
\usepackage{microtype}
\usepackage{letterbib}
\usepackage{natbib}
\frenchspacing

\signature{Peter D Smits}
\address{Committee on Evolutionary Biology \\ University of Chicago \\
1025 E. 57th Street \\ Culver Hall 402 \\ Chicago, IL 60637 \\
psmits@uchicago.edu}

\begin{document}
\begin{letter}{Editor \\ \textit{Evolution}}
  \opening{Dear Editor,}

  Please find enclosed the manuscript entitled ``The interplay between extinction intensity and selectivity: correlation in trait effects on taxonomic survival.'' In this study I analyzed the effect of biological traits on brachiopod genus durations in order to understand how increases or decreases in extinction intensity influence the selectivity, or importance, of these traits. The biological traits analyzed included body size, geographic range size, and environmental preference.
  
  While the effect of geographic range on extinction risk is well documented, how other traits may increase or decrease extinction risk is less well known. I analyze patterns of Paleozoic brachiopod genus durations and their relationship to geographic range, affinity for epicontinental seas versus open ocean environments, and body size. Using a hierarchical Bayesian approach, I also model the interaction between the effects of biological traits and a taxon's time of origination. 

  I find support for a ``survival of the generalists'' scenario for most of the Paleozoic, though there are times where this relationship is absent or even reversed. Importantly, I find evidence that as baseline extinction risk increases, the effect of geographic range increases but the effect of environmental preference tends to decrease/become more linear. This is consistent with previous hypotheses about the relative effects of biological traits and the selectivity of extinction \cite{Jablonski1987,Raup1991b}. Additionally, I find strong evidence for correlation between the effects of geographic range and the non-linear aspect of environmental preference which may help explain the processes underlying this pattern. 
  
  Given these results and the correlation between effects in particular, I hypothesize that because taxa with large geographic ranges encompass more possible environments, when extinction intensity is high there is little coherent difference in the environmental preference among the surviving taxa. The intensity decreases the selectivity such that the effect of environmental preference is effectively washed out by the strength of the effect of geographic range. This builds upon similar hypotheses developed by Raup \cite{Raup1991b,Raup1994} by incorporating a specific macroevolutionary mechanism based on the correlations between baseline extinction risk, geographic range size, and environmental preference.
  
  If accepted, all data and code necessary to duplicate this analysis will be made available on DRYAD.

  Possibly appropriate reviewers include Paul Harnik (paul.harnik@fandm.edu, Franklin and Marshall College), Steve Wang (scwang@swarthmore.edu, Swarthmore College), and Carl Simpson (simpsoncg@si.edu, Smithsonian Institution).
  
  Thank you for considering our work. Please send all correspondence regarding this manuscript to me via my email address (psmits@uchicago.edu).

  \closing{Sincerely,}

  \encl{Article; supplementary text, figures, tables.}

\end{letter}
\bibliographystyle{plain}
\bibliography{newbib}
\end{document}

